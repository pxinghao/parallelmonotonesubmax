\documentclass{article}

\usepackage{amsmath}
\usepackage{amssymb}
\usepackage{natbib}

\usepackage{color}

% Commenting system
\newcommand{\Comments}{1}
\newcommand{\note}[2]{\ifnum\Comments=1\textcolor{#1}{#2}\fi}
\newcommand{\xinghao}[1]{\note{red}{[XP: #1]}}
\newcommand{\joey}[1]{\note{blue}{[JG: #1]}}
\newcommand{\stef}[1]{\note{green}{[SJ: #1]}}

\begin{document}

\section{Distributed / parallel algorithms}

\subsection{Monotone}

\begin{description}
\item[\citet{Mirzasoleiman2013}***] Divide-and-conquer, bounds are only good with other assumptions (locality, sampling for representation)
\item[\citet{kumar13}***] MapReduce in several rounds. Trick is an $\epsilon$-greedy subroutine, and greedy wrt a current set (? check that)
\item[\citet{wei14}***] key idea is a modular (conditional) approximation of the submodular function. \xinghao{From my cursory reading, I think this is an approximation algorithm that builds on lazy greedy, and is complementary to distributive algorithms. It is not, by itself, a distributed algorithm.}
\end{description}

\subsection{Set cover as a special case}

\begin{description}
\item[\citet{blelloch12}] seems to a follow-up of \citep{blelloch11}
\item{\citet{abrams04}] set $k$-cover
\item[\citet{koufog}]
\item[\citet{chierichetti10}] MAP-reduce
\end{description}

\subsection{Non-monotone}



\section{Streaming/online algorithms}

See also Section 4.4 in Andreas Krause's survey for an overview.

\subsection{Monotone}

\begin{description}
\item[\citet{kapralov13}**] prove that no online algorithm (even randomized, against an oblivious adversary) is better than $1/2$-competitive for welfare maximization with coverage valuations. Also prove that Greedy in a stochastic setting with i.i.d. items and valuations satisfying diminishing returns is $(1-1/e)$-competitive.
\item[\citet{buchbinder12}] not sure how relevant
\item[\citet{feldman11sec}***] 
\end{description}


\subsection{Non-monotone}

\begin{description}
\item[\cite{bateni10}***] (non-monotone) submodular secretary problem, with matroid constraints; $O(l \log^2 r)$ ($l$ matroids, each of rank $r$ or less); also knapsack constraints and subadditive utility functions.
\item[\citet{gupta10}***] 
\end{description}

\subsection{Linear opt over matroid}
\begin{description}
\item[\citet{babaioff}]
\item[\citet{ogharan11}]
\item[\citet{imXX}] 
\end{description}

\subsection{Other maybe related}
\begin{description}
\item[\citet{feldman12parallel}] 
\end{description}


\subsection{Online learning}

That means multiple repetitions of a ``game'' to ``learn'' a good solution.

\begin{description}
\item[\citet{ross13}] Bandit greedy with no regret. (essentially learning a list). Intuitively, I think this is related to the idea of viewing greedy as optimizing a linear (subgradient-based) approximation \citep{iyer13icml}
\item[\citet{streeter09}] 
\end{description}


\section{Other algorithms}

\begin{description}
\item[\citet{badan14}] The basis of \cite{wei14}. 
\end{description}

\section{Exploiting additional properties}

\subsubsection{Curvature}
\begin{description}
\item[\citet{sviridenko13}]
\item[\citet{wei14}] (I think, to be checked)
\item[\citet{iyer13curv}] (not max; but min and learning)
\item[\citet{conforti84}] improved bounds for the greedy algorithm
\item[\citet{vondrak10}] 
\end{description}

\subsubsection{Approximate function (specialized)}
\begin{description}
\item[\citet{borgs14}] 
\end{description}

\subsubsection{Locality, repetition}
\begin{description}
\item[\citet{Mirzasoleiman2013}] 
\end{description}


\section{Examples of large scale applications or of speeding up functions}

\begin{description}
\item[\citet{borgs14}] Expressing diffusion functions on networks via hypergraph (covers). Factor of $O(1 - 1/e - \epsilon)$ in time $O((m+n)\epsilon^{-3} \log n)$. Also some bounds for ``early stopping''.
\end{description}




\section{Other ideas}

\begin{itemize}
\item Graph sparsification: OCC-ifying \citet{kelner11}
\end{itemize}


\bibliographystyle{abbrvnat}
\bibliography{references_arxiv}

\end{document}
